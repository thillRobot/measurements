% 
% Lecture Template for ME3023 -  Measurements in Mechanical Systems - Tennessee Technological University
%
% Spring 2020 - Summer 2020
% Tristan Hill, May 07, 2020 - June 12, 2020 - July 08, 2020
% Lecture Module - Electrical Instruments
% Topic 2 - Performing Analog Measurements
%

\documentclass[fleqn]{beamer} % for presentation (has nav buttons at bottom)

%\usepackage{/home/thill/courses/measurements/lectures/measurements_lectures}
\usepackage{/home/tntech.edu/thill/courses/measurements/lectures/measurements_lectures}

\author{ME3023 - Measurements in Mechanical Systems} % original formatting from Mike Renfro, September 21, 2004

\newcommand{\MNUM}{5\hspace{2mm}} % Module number
\newcommand{\TNUM}{1\hspace{2mm}} % Topic number 
\newcommand{\moduletitle}{Electrical Instruments}
\newcommand{\topictitle}{Performing Analog Measurements} 

\newcommand{\sectiontitleI}{Safety and Electricity }
\newcommand{\sectiontitleII}{Using a Digital Multimeter}
\newcommand{\sectiontitleIII}{Measuring Voltage and Current}
\newcommand{\sectiontitleIV}{Measuring Resistance and Continuity}
\newcommand{\sectiontitleV}{Using an Oscilloscope}

% custom box
\newsavebox{\mybox}

\title{Lecture Module - \moduletitle}

\date{Mechanical Engineering\vspc Tennessee Technological University}

\begin{document}
	
	\lstset{language=MATLAB,basicstyle=\ttfamily\small,showstringspaces=false}
	
	\frame{\titlepage \center\begin{framed}\Large \textbf{Topic \TNUM - \topictitle}\end{framed} \vspace{5mm}}
	
	
	% Section 0: Outline
	\frame{
		\large \textbf{Topic \TNUM - \topictitle} \vspace{3mm}\\
		
		\begin{multicols}{2}

			\begin{itemize}
				
				\item \sectiontitleI    \vspc % Section I
				\item \sectiontitleII 	\vspc % Section II
				\item \sectiontitleIII 	\vspc %Section III
				\item \sectiontitleIV 	\vspc %Section IV
				
			\end{itemize}
			
			\includegraphics[scale=.10]{YX360TRF_Sanwa.jpg}

		\end{multicols}

{\tiny \href{https://commons.wikimedia.org/wiki/File:YX360TRF(Sanwa).JPG}{Sanwa YX360TRF} }

	}


\section{\sectiontitleI}

% Section I - Frame I:
	\frame{  \small
		\frametitle{\sectiontitleI}

		Complete the saftey training modules to learn more. 

	}


\section{\sectiontitleII}

% Section II - Frame I:
	\begin{frame}  \scriptsize
		\frametitle{\sectiontitleII}
		Benchtop Multimeter
			 
		\begin{itemize}
			\item DMM 
			\item Interface varies by manufacturer
			\item Features vary by model
		\end{itemize} 
		
		\begin{multicols}{2}
		
		\vspace{5mm}
		\includegraphics[scale=.25]{dmm_benchtop.jpeg}
		\tiny{Image: Fluke Multimeter}	 
		\vspace{50mm}

		\includegraphics[scale=.13]{bk_2831e.jpg}
		\vspace{10mm}
		\tiny{Image: Bk Precision Multimeter }	
		\vspace{10mm}

		\end{multicols}	


	\end{frame}

	% Section II - Frame I:
	\begin{frame}  \scriptsize
		\frametitle{\sectiontitleII}
		Handheld Multimeter

		\begin{multicols}{3}
			 
		\includegraphics[scale=0.04]{Fluke_115_multimeter.jpeg}
		\tiny{Image:\href{https://commons.wikimedia.org/wiki/File:Fluke_115_multimeter.jpg}{}}	 

		\includegraphics[scale=.15]{Digital_Multimeter_Aka.jpg}
		\tiny{Image: \href{https://commons.wikimedia.org/wiki/File:Digital_Multimeter_Aka.jpg}{}}
		
		\includegraphics[scale=.20]{Clampmeter_Fluke_337}
		\tiny{Image: \href{https://commons.wikimedia.org/wiki/File:Clampmeter_Fluke_337.jpg}{}}

		\end{multicols}	


	\end{frame}

\section{\sectiontitleIII}

% Section II - Frame I:
\frame{  \small
\frametitle{\sectiontitleIII}

 \underline{{\bf \large Voltage}} \vspace{10mm}\\ 
 
 \includegraphics[scale=.13]{bk_2831e.jpg}
 \vspace{10mm}
 \tiny{Image: Bk Precision Multimeter }
 
 \textbf{ \href{https://www.fluke.com/en-us/learn/best-practices/test-tools-basics/digital-multimeters/how-to-measure-dc-voltage-with-a-digital-multimeter}{Read about measuring DC voltage} }

\textbf{ \href{https://www.fluke.com/en-us/learn/best-practices/test-tools-basics/digital-multimeters/how-to-measure-ac-voltage-with-a-digital-multimeter}{ or AC voltage} } 

}


% Section III - Frame II:
\frame{  \small
\frametitle{\sectiontitleIII}
 
 \underline{{\bf Current}} \vspace{10mm}\\  
 
 \includegraphics[scale=.13]{bk_2831e.jpg}
 \vspace{10mm}
 \tiny{Image: Bk Precision Multimeter }
\textbf{ \href{https://www.fluke.com/en-us/learn/best-practices/test-tools-basics/digital-multimeters/how-to-measure-current-with-a-digital-multimeter-plus-clamp-accessory}{Read about measuring current (with clamp)} }


}

\section{\sectiontitleIV}

% Section IV - Frame I:
\frame{  \small
\frametitle{\sectiontitleIV}

 \underline{{\bf \large Resistance}} \vspace{10mm}\\ 
 
  \includegraphics[scale=.13]{bk_2831e.jpg}
 \vspace{10mm}
 \tiny{Image: Bk Precision Multimeter }
 
  \textbf{ \href{https://www.fluke.com/en-us/learn/best-practices/test-tools-basics/digital-multimeters/how-to-measure-resistance}{Read about measuring resistance.} }


}

% Section IV - Frame II:
\frame{  \small
\frametitle{\sectiontitleIV}

 \underline{{\bf \large Continuity}} \vspace{10mm}\\ 
 
  \includegraphics[scale=.13]{bk_2831e.jpg}
 \vspace{10mm}
 \tiny{Image: Bk Precision Multimeter }

  \textbf{ \href{https://www.fluke.com/en-us/learn/best-practices/test-tools-basics/digital-multimeters/how-to-test-for-continuity-with-a-digital-multimeter}{Read about measuring continuity} } 


}


\end{document}
%\begin{document}
%
%\textbf{ \LARGE ME3023 Lecture -  Chapter \NUM \\\\ \hspace*{5mm} Analog Electrical Devices
%and Measurements} \\\\
%\textbf{ \hspace*{5mm}\underline{Theory and Design for Mechanical Measurements}\vspace{1mm}\\ 
%                \hspace*{5mm} 5th ed. by Richard Figliola and Donald Beasley}\vspace{3mm}\\
%\textbf{ \hspace*{5mm}Tristan Hill - Tennessee Technological University - Fall 2019} \vspace{3mm}\\
%
%\begin{itemize}
%
%
%	\item \textbf{ \LARGE 6.1 -  Introduction  } \\\\
%	
%		\textbf{\Large "... basic electrical analog devices used with analog signals or to display signals in an analog form ..."} \\\\
%		 
%		\textbf{\Large " ... Information is often transferred between stages of a
%measurement system as an analog electrical signal. This signal typically originates from the
%measurement of a physical variable using a fundamental electromagnetic or electrical phenomenon
%and then propagates from stage to stage of the measurement system ... "} \\\\
%		\textbf{\Large "...  Within a signal chain, it is common to find digital and analog electrical devices being used together ...} \\
%
%\newpage		
%		 \textbf{\Large Upon completion of this chapter, the reader will be able to}\\\\
%		
%		\begin{itemize}
%			\item \textbf{\Large understand the principles behind common analog voltage and current measuring
%devices} \vspace{10mm}
%			\item \textbf{\Large understand the operation of balanced and unbalanced resistance bridge circuits} \vspace{10mm}
%			\item \textbf{\Large define, identify, and minimize loading errors}  \vspace{10mm}
%\item \textbf{\Large  understand the basic principles involved in signal conditioning, especially filtering and
%amplification, and }  \vspace{10mm}
%\item \textbf{\Large  apply proper grounding and shielding techniques in measuring system hookups.}  \vspace{10mm}
%		\end{itemize}
%		
%		\newpage
%	
%	\item \textbf{ \LARGE 6.1.0 -  BASIC ANALOG COMPONENTS : } \\
%\newpage
%	\item \textbf{ \LARGE 6.1.1 -  IMPORTANT QUANTITIES : } \\
%\newpage
%\item \textbf{ \LARGE 6.1.2 -  GOVERNING MATHEMATICS : } \\
%\newpage
%	\item \textbf{ \LARGE 6.1.3 -  PERFORMING ANALOG MEASUREMENTS: } \\
%\begin{itemize}
%
%		\item \textbf{ \LARGE \href{https://www.fluke.com/en-us/learn/best-practices/test-tools-basics/digital-multimeters/how-to-measure-resistance}{Measuring Resistance} } \vspace{130mm}\\ \includegraphics[scale=0.25]{lecture1_fig10.png} \\
%\newpage
%		\item \textbf{ \LARGE \href{https://www.fluke.com/en-us/learn/best-practices/test-tools-basics/digital-multimeters/how-to-test-for-continuity-with-a-digital-multimeter}{Continuity} } \vspace{130mm}\\ \includegraphics[scale=0.25]{lecture1_fig10.png} \\
%\newpage
%\item \textbf{ \LARGE \href{https://www.fluke.com/en-us/learn/best-practices/test-tools-basics/digital-multimeters/how-to-measure-capacitance-with-a-digital-multimeter}{Capacitance} } \vspace{130mm}\\ \includegraphics[scale=0.25]{lecture1_fig10.png} \\
%\newpage
%\item \textbf{ \LARGE \href{https://www.fluke.com/en-us/learn/best-practices/test-tools-basics/digital-multimeters/how-to-measure-dc-voltage-with-a-digital-multimeter}{DC voltage} }\vspace{130mm}\\ \includegraphics[scale=0.25]{lecture1_fig10.png} \\
%\newpage
%\item \textbf{ \LARGE \href{https://www.fluke.com/en-us/learn/best-practices/test-tools-basics/digital-multimeters/how-to-measure-ac-voltage-with-a-digital-multimeter}{AC voltage} }\vspace{130mm}\\ \includegraphics[scale=0.25]{lecture1_fig10.png} \\
%\newpage
%\item \textbf{ \LARGE \href{https://www.fluke.com/en-us/learn/best-practices/test-tools-basics/digital-multimeters/how-to-measure-current-with-a-digital-multimeter-plus-clamp-accessory}{Current (with clamp)} }\vspace{130mm}\\ \includegraphics[scale=0.25]{lecture1_fig10.png} \\
%
%\end{itemize}
%
%	\newpage
%	\item \textbf{ \LARGE 6.2 -  ANALOG DEVICES: CURRENT } \\\\
%	\begin{itemize}
%		\item \textbf{ \LARGE Direct Current  } \\\\
%
%\includegraphics[scale=0.5]{lecture1_fig6_1.png} \\
%\includegraphics[scale=0.5]{lecture1_fig6_3.png}
%
%\scalebox{1}{$F=IlB$} \\
%\scalebox{1}{$\bf{F}=Il\hat{\bf{k}}\times\bf{B}$} \\
%\scalebox{1}{$T_{mu}=NIABsin{\alpha}$} \\
%
%\newpage
%		\item \textbf{ \LARGE Alternating Current} \\\\
%
%	\end{itemize}
%
%	\newpage
%	\item \textbf{ \LARGE 6.3 - ANALOG DEVICES: VOLTAGE} \\
%	\begin{itemize}
%		\item \textbf{ \LARGE Analog Voltage Meters  } \\
%\includegraphics[scale=0.75]{lecture1_fig6_6.png} \\
%		\item \textbf{ \LARGE Oscilloscope} \\
%\includegraphics[scale=0.75]{lecture1_fig6_7.png} \\
%		\item \textbf{ \LARGE Potentiometer} \\\\
%		\includegraphics[scale=0.75]{lecture1_fig6_10.png} \\
%		\begin{itemize}
%			\item \textbf{ \LARGE Voltage Divider Circuit} \\\\
%
%
%
%			\item \textbf{ \LARGE Potentiometer Instruments} \\\\
%	
%		\end{itemize}
%	\end{itemize}
%
%	\newpage
%	\item \textbf{ \LARGE 6.4 -  ANALOG DEVICES: Resistance } \\\\
%	\begin{itemize}
%		\item \textbf{ \LARGE Ohmmeter Circuits  } \\\\
%		\item \textbf{ \LARGE Bridge Circuits} \\\\
%\includegraphics[scale=0.75]{lecture1_fig6_13.png} \\
%
%\newpage
%		\item \textbf{ \LARGE Null Method} \\\\
%		\item \textbf{ \LARGE Deflection Method} \\\\
%
%	\end{itemize}
%	
%\newpage
%	\item \textbf{ \LARGE 6.5 -  LOADING ERRORS AND IMPEDANCE MATCHING } \\\\
%	\begin{itemize}
%		\item \textbf{ \LARGE Loading Errors for Voltage-Dividing Circuit  } \\\\
%		\item \textbf{ \LARGE Interstage Loading Errors} \\\\
%
%	\end{itemize}
%
%\newpage
%	\item \textbf{ \LARGE 6.6 -  ANALOG SIGNAL CONDITIONING: AMPLIFIERS } \\\\
%	\begin{itemize}
%		\item \textbf{ \LARGE Operational Amplifiers} \\\\
%		\includegraphics[scale=0.6]{lecture1_fig6_19.png} \\\\
%
%\item \textbf{ \LARGE Operational Amplifiers} \\\\
%		\includegraphics[scale=0.6]{lecture1_fig6_20.png} \\
%	\end{itemize}	
%\newpage
%
%	\item \textbf{ \LARGE 6.7 -  ANALOG SIGNAL CONDITIONING: SPECIAL-PURPOSE CIRCUITS } \\\\
%	\begin{itemize}
%		\item \textbf{ \LARGE Analog Voltage Comparator} \\\\
%		\item \textbf{ \LARGE Sample-and-Hold Circuit} \\\\
%		\item \textbf{ \LARGE Charge Amplifier} \\\\
%		\item \textbf{ \LARGE 4-20mA Current Loop} \\\\
%		\item \textbf{ \LARGE Multivibrator and Flip-Flop Circuits} \\\\
%	\end{itemize}	
%\newpage
%	\item \textbf{ \LARGE 6.8 -  ANALOG SIGNAL CONDITIONING: Filters } \\\\
%
%		\includegraphics[scale=0.45,angle=0,origin=c]{lecture1_fig6_27.png} \\
%		\includegraphics[scale=0.45]{lecture1_fig6_28.png} \\
%	\begin{itemize}
%		\item \textbf{ \LARGE Butterworth Filter Design} \\\\
%		\item \textbf{ \LARGE Improved Butterworth Filter Designs} \\\\
%		\item \textbf{ \LARGE Bessel Filter Design} \\\\
%		\item \textbf{ \LARGE Active Filters} \\\\
%	\end{itemize}	
%\newpage
%	\item \textbf{ \LARGE 6.9 GROUNDS, SHIELDING, AND CONNECTING WIRES } \\\\
%	\begin{itemize}
%		\item \textbf{ \LARGE Ground and Ground Loops} \\\\
%		\item \textbf{ \LARGE Connecting Wires} \\\\
%	\end{itemize}	
%\newpage
%	\item \textbf{ \LARGE 6.10 SUMMARY } \\\\
%	\begin{itemize}
%		\item \textbf{ \LARGE woop woop} \\\\
%	\end{itemize}	
%
%		%\includegraphics[scale=1]{lecture1_fig1.png}
%		
%\end{itemize}
%
%
%	
%
%\end{document}
%
%
%
