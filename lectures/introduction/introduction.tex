% Lecture Template for ME3023 -  Measurements in Mechanical Systems - Tennessee Technological University
% Spring 2020 - Summer 2020 - Fall 2020 - Spring 2021 - Summer 2021
% Tristan Hill, May 07, 2020 - June 12, 2020 - July 08, 2020 - Novemeber 02, 2020 - March 28, 2021 - May 25, 2021 - August 21, 2022

% Fall 2023 - condensing and streamlining lectures by combining topics into a single PDF under the module name
%			  this will simplify file and link management as well as make lectures easier to use in class
%			- added image/ to clean directory and reduce redundancy, specific to module for now  

% Module Name: - Introduction
% Topic 1 - General Measurement System
% Topic 2 - Types of Variables 
% Topic 3 - Experimental Test Plan  ! needs improvement ! 

\documentclass[fleqn]{beamer} % for presentation (has nav buttons at bottom)

%\usepackage{/home/thill/courses/measurements_lectures/measurements_lectures}
\usepackage{/mnt/c/Users/thill/Documents/courses/measurements/lectures/measurements_lectures}

\author{ME3023 - Measurements in Mechanical Systems} % original formatting from Mike Renfro, September 21, 2004

\newcommand{\MNUM}{1\hspace{2mm}} % module number 
\newcommand{\moduletitle}{Introduction}
%\newcommand{\topictitle}{General Measurement System} 

\newcommand{\sectionItitle}{General Measurement System}
\newcommand{\sectionIItitle}{Types of Variables}
\newcommand{\sectionIIItitle}{Experimental Test Plan}
\newcommand{\sectionIVtitle}{Numbers and Storage}

\newcommand{\sectionIsubsectionItitle}{Definition of a Measurement}
\newcommand{\sectionIsubsectionIItitle}{Measurement System Stages}
\newcommand{\sectionIsubsectionIIItitle}{Brainstorming Activity}
\newcommand{\sectionIsubsectionIVtitle}{Examples in Mechcanical Engineering}

% custom box
\newsavebox{\mybox}

\title{Lecture Module - \moduletitle}

\date{Mechanical Engineering\vspc Tennessee Technological University}

\begin{document}

	\lstset{language=MATLAB,basicstyle=\ttfamily\small,showstringspaces=false}

	\frame{\titlepage \center\begin{framed}\Large \textbf{Topic \MNUM - \moduletitle}\end{framed} \vspace{5mm}}

	% Module Outline
	\begin{frame} 
		\large \textbf{Module \MNUM - \moduletitle} \vspace{3mm}\\

		\begin{itemize}
			\item Topic 1 - \hyperlink{sectionI}{\sectionItitle} \vspc % section I
			\item Topic 2 - \hyperlink{sectionII}{\sectionIItitle} \vspc % section II
			\item Topic 3 - \hyperlink{sectionIII}{\sectionIIItitle} \vspc % section III
			\item Topic 4 - \hyperlink{sectionIV}{\sectionIVtitle} \vspc % section IV
		\end{itemize}

	\end{frame}




	% section I
	\section{\sectionItitle}\label{sectionI}

		% section I Outline
		\begin{frame} 
			\large \textbf{Topic 1 - \sectionItitle} \vspace{3mm}\\

			\begin{itemize}
				\item \hyperlink{sectionIsubsectionI}{\sectionIsubsectionItitle} \vspc %  section I subsection I
				\item \hyperlink{sectionIsubsectionII}{\sectionIsubsectionIItitle} \vspc % section I subsection II
				\item \hyperlink{sectionIsubsectionIII}{\sectionIsubsectionIIItitle} \vspc % section I subsection III
				\item \hyperlink{sectionIsubsectionIV}{\sectionIsubsectionIVtitle} \vspc % section I subsection IV
			\end{itemize}
		\end{frame}
		
		% section I subsection I 
		\subsection{\sectionIsubsectionItitle}\label{sectionIsubsectionI}

			\begin{frame}
				\frametitle{\sectionIsubsectionItitle}

				\large{``A {\bf \BL measurement} is an act of assigning a specific value to a physical variable.''} \vspc
				{\tiny Text: Theory and Design of Mech. Meas.}

			\end{frame}

		% section I subsection II
		\subsection{\sectionIsubsectionIItitle}\label{sectionIsubsectionII}

			\begin{frame}
				\frametitle{\sectionIsubsectionIItitle}

				\includegraphics[scale=.3]{images/measurement_stages.png} \\
				{\tiny Image: Theory and Design of Mech. Meas.}
			\end{frame}

			\begin{frame}
				\frametitle{Sensor-Transducer Stage}
				a {\PR sensor}, a physical element that employs some natural phenomenon... ...to sense the variable being measured
				\includegraphics[scale=0.20]{images/sensor_stage.png}\includegraphics[scale=0.20]{images/sensor_transducer_stage.png}\\

				A {\GR transducer} converts the sensed information into a detectable signal \\
				{\tiny Text, Image: Theory and Design of Mech. Meas.}
			\end{frame}

			\begin{frame}
				\frametitle{Signal Conditioning Stage}

				What is the the definition of {\BL signal}? \vspc

				\begin{multicols}{2}
				\includegraphics[scale=0.18]{images/signal_noise.png}

				\begin{itemize}
				\item Filtering
				\item Amplification
				\item Attenuation
				\item Excitation 
				\item Linearization
				\item Electrical Isolation
				\item Surge Protection
				\end{itemize}

				\end{multicols}

				{\tiny Image: Theory and Design of Mech. Meas.}
			\end{frame}

			\begin{frame}
				\frametitle{Output Stage}
				The {\BR output stage} indicates or records the value measured. This might be a simple readout
				display, a marked scale, or even a recording device such as a computer disk drive.

				\includegraphics[scale=0.25]{images/bulb_thermometer.png} \hspace{10mm}\includegraphics[scale=0.3]{images/thermocouple.jpg}

				{\tiny Image: Theory and Design of Mech. Meas. \hspace{20mm} Image: \href{https://en.wikipedia.org/wiki/Thermocouple}{Wikipedia} }
			\end{frame}

		% section I subsection III
		\subsection{\sectionIsubsectionIIItitle}\label{sectionIsubsectionIII}
			\begin{frame} 
				\frametitle{\sectionIVtitle}
				
				\begin{multicols}{2}
					\tiny

					{\bf Activity:} Team Brainstorm

					{\bf Duration:} $\sim 10$ minutes

					{\bf Groups:} 2-3 members

					\includegraphics[scale=0.5]{images/Brainstorm_room.png}

					{\bf Topic:} Remote Probe Concept
					\begin{itemize}
						\item You are designing a remote probe to inspect an environment which can only be accessed from above. 
						\item The goal is to collect as much information as possible from the environment to prepare for a robotic maintinence task. 
			        \end{itemize}

			    \end{multicols}	

			    {\bf Requirements:}	
				\begin{itemize}
					\item Probe must enter environment through hole $\sim 100mm$ wide 
					\item Probe must exit through same hole leaving nothing behind
					\item The alllowable EFI and RFI is limited. No wifi communication is available \vspace{4mm}
				\end{itemize}

				{\bf Deliverable:} Submit a copy of your team brainstorming notes including text, images, and diagrams to the activity assignment on ilearn. Include names of all team members.

			\end{frame}	

		% section I subsection IV
		\subsection{\sectionIsubsectionIVtitle}\label{sectionIsubsectionIV}	

			\begin{frame}[label=sectionIV]
				\frametitle{\sectionIVtitle}
				\href{https://events-platform.asmeconferences.org/event/idetc-cie-2022/planning/UGxhbm5pbmdfOTcxMjI2}{IDETC2022-96785: Development of an Instrumented Rear Suspension to Measure the Tire Forces of a Race Car During Track Driving}\vspace{5mm}\\

				\includegraphics[scale=0.125]{images/IDETC_technical_session.png}

			\end{frame}

			\begin{frame}[label=sectionIV]
				\frametitle{\sectionIVtitle}

				\href{https://events-platform.asmeconferences.org/event/idetc-cie-2022/planning/UGxhbm5pbmdfOTcxMzIx}{IDETC2022-91154: Photometric Stereo Enhanced Light Sectioning Measurement for Microtexture Road Profiling}\vspace{5mm}\\

				\includegraphics[scale=0.125]{images/IDETC_technical_session.png}
			 
			\end{frame}

			\begin{frame}[label=sectionIV]
				\frametitle{\sectionIVtitle}
				\href{https://events-platform.asmeconferences.org/event/idetc-cie-2022/planning/UGxhbm5pbmdfOTcxNDk4}{IDETC2022-90082: Automated Weld Path Generation Using Random Sample Consensus and Iterative Closest Point Workpiece Localization}\vspace{5mm}\\

				\includegraphics[scale=0.125]{images/IDETC_technical_session.png}

			\end{frame}	


	% Section II
	\section{\sectionIItitle}\label{sectionII}

		\begin{frame}[label=sectionII]
			\frametitle{\sectionIItitle}

		\end{frame}


	% Section III
	\section{\sectionIIItitle}\label{sectionIII}

		\begin{frame}[label=sectionIII] 
			\frametitle{\sectionIVtitle}


		\end{frame}


	% Section IV
	\section{\sectionIVtitle}\label{sectionIV}

		\begin{frame}[label=sectionIV]
			\frametitle{\sectionIVtitle}

		\end{frame}

\end{document}





