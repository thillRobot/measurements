% !TeX document-id = {3ffda977-020f-403a-a748-6559a1e64ed1}
% !TeX TXS-program:compile = txs:///pdflatex/[--shell-escape]
% % !TEX TS-program = xelatex
% !TEX encoding = UTF-8 Unicode


% Spring 2020 - Summer 2020 - Fall 2020 - SPring 2021
% Tristan Hill, May 07, 2020 - June 12, 2020 - July 08, 2020 - Novemeber 02, 2020 - March 28, 2021
% Module Name: - Data Acquisition
% Topic 1 - Analog to Digital Conversion

\documentclass[fleqn]{beamer} % for presentation (has nav buttons at bottom)

%\usepackage{/home/tntech.edu/thill/courses/measurements/lectures/measurements_lectures}
\usepackage{/mnt/c/Users/thill/Documents/courses/measurements/lectures/measurements_lectures}

\author{ME3023 - Measurements in Mechanical Systems} % original formatting from Mike Renfro, September 21, 2004

\newcommand{\TNUM}{1\hspace{2mm}} % Topic number - Topic numbers are going to stay
\newcommand{\moduletitle}{Data Acquisition}
\newcommand{\topictitle}{Analog to Digitial Conversion} 

\newcommand{\sectiontitleI}{DAQ and Computer Storage}
\newcommand{\sectiontitleII}{Types of Integers}
\newcommand{\sectiontitleIII}{Integers vs. Floating Point Numbers}
\newcommand{\sectiontitleIV}{Analog to Digital Conversion}
\newcommand{\sectiontitleV}{Activity: ADC Resolution Calculation}

\newcommand{\btVFill}{\vskip0pt plus 1filll}


% custom box
\newsavebox{\mybox}

\title{Lecture Module - \moduletitle}

\date{Mechanical Engineering\vspc Tennessee Technological University}

\begin{document}

\lstset{language=MATLAB,basicstyle=\ttfamily\small,showstringspaces=false}

\frame{\titlepage \center\begin{framed}\Large \textbf{Topic \TNUM - \topictitle}\end{framed} \vspace{5mm}}

% Section 0: Outline
\frame{
\large \textbf{Topic \TNUM - \topictitle} \vspace{3mm}\\

\begin{itemize}

	\item \sectiontitleI    \vspc % Section I
	\item \sectiontitleII 	\vspc % Section II
	\item \sectiontitleIII 	\vspc %Section III
	\item \sectiontitleIV 	\vspc %Section IV
  \item \sectiontitleV 	\vspc %Section IV

\end{itemize}

}

% Section I:
\section{\sectiontitleI}

	% Section I - Frame I:
	\begin{frame}[label=sectionI] \small
		\frametitle{\sectiontitleI}

		
		A data acquisition system is the portion of a measurement system that quantifies and stores data. - {\tiny Theory and Design of Mechanical Measurements}

		\begin{multicols}{2}
		\includegraphics[scale=.30]{cartesian_6x6_B.png} 
		

		\end{multicols}
		\btVFill
		\tiny{Image: T.Hill}
	\end{frame}
	
\section{\sectiontitleII}	




% Section II - Frame I
\begin{frame}[label=sectionII] \small
\frametitle{\sectiontitleII}
\bigskip

\begin{multicols}{2}
\begin{tabular}{|r|r|r|} \hline
	Binary 	& Decimal 	& Hexadecimal \\ \hline
	0		& 0			& 0 		\\ \hline	
	1		& 1			& 1 		\\ \hline
	10		& 2			& 2 		\\ \hline
	11		& 3			& 3 		\\ \hline
	100		& 4			& 4 		\\ \hline
			& 5			& 5 		\\ \hline
			& 6			& 6 		\\ \hline
		    & 7			& 7 		\\ \hline
			& 8			& 8 		\\ \hline
			& 9			& 9 		\\ \hline
			& 10		& A 		\\ \hline
			& 11		& B 		\\ \hline
\end{tabular}

\begin{tabular}{|r|r|r|} \hline
	Binary 	& Decimal 	& Hexadecimal \\ \hline
			& 12		& C 		\\ \hline	
			& 13		& D 		\\ \hline
			& 14		& E 		\\ \hline
			& 15		& F 		\\ \hline
			& 16		&  		\\ \hline
			& 17		&  		\\ \hline
			& 18		&  		\\ \hline
			& 19		&  		\\ \hline
			& 20		&  		\\ \hline
			& 21		&  		\\ \hline
			& 22	    &  		\\ \hline
			& 23	    &  		\\ \hline
\end{tabular}
\end{multicols}

\btVFill
\tiny{some reference}		

\end{frame}

% Section II - Frame II
\begin{frame}[label=sectionII] \small
\frametitle{\sectiontitleII}
\bigskip

\begin{multicols}{2}
\begin{tabular}{|r|r|r|} \hline
	Binary 	& Decimal 	& Hex. \\ \hline
	0		& 0			& 0 		\\ \hline	
	1		& 1			& 1 		\\ \hline
	10		& 2			& 2 		\\ \hline
	11		& 3			& 3 		\\ \hline
	100		& 4			& 4 		\\ \hline
	& 			&  		\\ \hline
	& 			&  		\\ \hline
	& 			&  		\\ \hline
	& 			&  		\\ \hline
	& 			&  		\\ \hline
	& 		&  		\\ \hline
	& 		&  		\\ \hline
\end{tabular}

\begin{tabular}{|r|r|r|} \hline
	Binary\hspace{18mm} 	& Decimal 	& Hex. \\ \hline
	0		& 0			& 0 		\\ \hline	
	1		& 1			& 1 		\\ \hline
	10		& 2			& 2 		\\ \hline
	11		& 3			& 3 		\\ \hline
	100		& 4			& 4 		\\ \hline
	& 			&  		\\ \hline
	& 			&  		\\ \hline
	& 			&  		\\ \hline
	& 			&  		\\ \hline
	& 			&  		\\ \hline
	& 		&  		\\ \hline
	& 		&  		\\ \hline
\end{tabular}
\end{multicols}

\btVFill
\tiny{some reference}	
\end{frame}


\section{\sectiontitleIII}

\begin{frame}[label=sectionIII] \small
 
  \frametitle{\sectiontitleIII}
  
  Standard definition of a floating point value in memory.\\	
  \includegraphics[scale=.10]{ieee754_bits.png}	
  \btVFill
	\href{https://en.wikipedia.org/wiki/IEEE_754}{Reference: wikipedia}
 
\end{frame}
	
% Section III - Frame I
\begin{frame}[label=sectionIII] \small
\frametitle{\sectiontitleIII}
\begin{multicols}{2}
	\underline{Integer} \vspace{20mm}\\
	
	\underline{Floating Point} \vspace{20mm}\\
	
\end{multicols}
\end{frame}

% Section III - Frame II
\begin{frame}[label=sectionIII] \small
\frametitle{\sectiontitleIII}
\begin{multicols}{2}
\underline{Integer} \vspace{20mm}\\
Pros:\vspace{10mm}\\
Cons:

\underline{Floating Point} \vspace{20mm}\\	
Pros:\vspace{10mm}\\
Cons:
\end{multicols}
\end{frame}


\section{\sectiontitleIV}	

% Section IV - Frame I
\begin{frame}[label=sectionIV] \scriptsize
\frametitle{\sectiontitleIV}
\bigskip

In electronics, an {\BL analog-to-digital converter} (ADC, A/D, or A-to-D) is a system that converts an analog signal, such as a sound picked up by a microphone or light entering a digital camera, into a digital signal. An ADC may also provide an isolated measurement such as an electronic device that converts an analog input voltage or current to a digital number representing the magnitude of the voltage or current. Typically the digital output is a two's complement binary number that is proportional to the input, but there are other possibilities.

\includegraphics[scale=.15]{WM_WM8775SEDS-AB.jpg} \hspace{5mm} \includegraphics[scale=.15]{ni_cdaq.jpg}

\btVFill
\tiny{\href{https://en.wikipedia.org/wiki/Analog-to-digital_converter}{wikipedia,} \href{https://en.wikipedia.org/wiki/Analog-to-digital_converter\#/media/File:WM_WM8775SEDS-AB.jpg}{image} }	
\end{frame}

% Section IV - Frame II
\begin{frame}[label=sectionIV] \scriptsize
\frametitle{\sectiontitleIV}
\bigskip

 It is important to realize the potential for data loss resulting in a reduced quality measurement based on the parameters of the analog to digital conversion process. This issue can occur when designing systems around a low-level analog to digital converter as well as when using high-end DAQ equippment.  

\includegraphics[scale=.2]{ADC_voltage_resolution.png}

\btVFill
\tiny{some reference}	
\end{frame}


\section{\sectiontitleV}	

% Section V - Frame I
\begin{frame}[label=sectionV] \small
\frametitle{\sectiontitleV}
\bigskip

\vspace{50mm}
\tiny{some reference}	
\end{frame}


\end{document}
