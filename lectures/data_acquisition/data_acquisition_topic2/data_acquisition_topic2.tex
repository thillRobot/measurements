% !TeX document-id = {3ffda977-020f-403a-a748-6559a1e64ed1}
% !TeX TXS-program:compile = txs:///pdflatex/[--shell-escape]
% % !TEX TS-program = xelatex
% !TEX encoding = UTF-8 Unicode


% Spring 2020 - Summer 2020 - Fall 2020 - SPring 2021
% Tristan Hill, May 07, 2020 - June 12, 2020 - July 08, 2020 - Novemeber 02, 2020 - March 28, 2021
% Module Name: - Data Acquisition
% Topic 2 - DAQ Hardware and Applications

\documentclass[fleqn]{beamer} % for presentation (has nav buttons at bottom)

%\usepackage{/home/thill/Documents/courses/measurements/lectures/measurements_lectures}
%\usepackage{/home/tntech.edu/thill/courses/measurements/lectures/measurements_lectures}
\usepackage{/mnt/c/Users/thill/Documents/courses/measurements/lectures/measurements_lectures}

\author{ME3023 - Measurements in Mechanical Systems} % original formatting from Mike Renfro, September 21, 2004

%\newcommand{\MNUM}{9\hspace{2mm}} % Module number - The module number should be phased out...
\newcommand{\TNUM}{2\hspace{2mm}} % Topic number - Topic numbers are going to stay
\newcommand{\moduletitle}{Data Acquisition}
\newcommand{\topictitle}{DAQ Hardware and Applications} 

\newcommand{\sectiontitleI}{Signal Types and DAQ}
\newcommand{\sectiontitleII}{Noise Considerations}
\newcommand{\sectiontitleIII}{Available Hardware}
\newcommand{\sectiontitleIV}{Software Integration}

\newcommand{\btVFill}{\vskip0pt plus 1filll}


% custom box
\newsavebox{\mybox}

\title{Lecture Module - \moduletitle}

\date{Mechanical Engineering\vspc Tennessee Technological University}

\begin{document}

\lstset{language=MATLAB,basicstyle=\ttfamily\small,showstringspaces=false}

\frame{\titlepage \center\begin{framed}\Large \textbf{Topic \TNUM - \topictitle}\end{framed} \vspace{5mm}}

% Section 0: Outline
\frame{
\large \textbf{Topic \TNUM - \topictitle} \vspace{3mm}\\

\begin{itemize}

	\item \sectiontitleI    \vspc % Section I
	\item \sectiontitleII 	\vspc % Section II
	\item \sectiontitleIII 	\vspc %Section III
	\item \sectiontitleIV 	\vspc %Section IV
%	\item \sectiontitleV 	\vspc %Section IV

\end{itemize}

}

% Section I:
\section{\sectiontitleI}

	% Section I - Frame I:
	\begin{frame}[label=sectionI] \scriptsize
	\frametitle{\sectiontitleI}
	
		{\it Most} data acquisition devices and systems measure and record {\BL analog} voltage signals and possibly additional signal types. Signal {\GR generation} may also be a feature on some systems. \vspace{5mm}\\


		A voltage signal requires a {\bf common} reference or {\bf ground}.	\vspace{2mm}\\

		Signal Sources:
		\begin{itemize}
			\item Grounded or Ground-Referenced	\vspace{2mm}\\

			\item Ungrounded or Floating \vspace{2mm}\\		

		\end{itemize}	
		\vspace*{5mm}

		Measurement (DAQ) Systems:
		\begin{itemize}
			\item Common Ground \vspace{2mm}\\

			\item Common Mode Voltage \vspace{2mm}\\

			\item Isolated Ground \vspace{2mm}\\

		\end{itemize}
		

		\btVFill
		\tiny{\href{https://www.ni.com/en/support/documentation/supplemental/06/grounding-considerations---intermediate-analog-concepts.html}{NI},}
		\tiny{\href{https://digilent.com/reference/daq-and-datalogging/documents/analog-input-signal-connections-1}{Digilent}}

	\end{frame}


	% Section I - Frame II:
	\begin{frame}[label=sectionI] \scriptsize
		\frametitle{\sectiontitleI}

		{\it Most} data acquisition devices and systems measure and record {\BL analog} voltage signals and possibly additional signal types. Signal {\GR generation} may also be a feature on some systems. \vspace{5mm}\\

		%\begin{multicols}{2}
		2 Major Configurations:
		\begin{itemize}
			\item
			\underline{Single-Ended Signals} \vspace{0mm}\\
			The signal is measured as a voltage between a {\PR single} conductor and the {\bf ground} which must be carried on a separate conductor or wire. \vspace{10mm}\\
		
			\item
			\underline{Double-Ended (Differential) Signals} \vspace{0mm}\\			
			The signal is measured as the {\PN difference} between two voltages ({\PN double}) carried on separate conductors, or wires. Typically a {\bf ground} is shared between the two devices requiring a third conductor. 
		\end{itemize}
	
		%\end{multicols}
		\btVFill
		\tiny{Read more here: \href{https://www.mccdaq.com/TechTips/TechTip-4.aspx}{MCCDAQ}}
	\end{frame}



	% Section I - Frame III:
	\begin{frame}[label=sectionI] \small
	\frametitle{\sectiontitleI}
		
		\begin{multicols}{2}
			\underline{Single-Ended Signals} \vspace{20mm}\\
			
			Pros:\vspace{10mm}\\
			Cons:
			
			\underline{Double-Ended Signals} \vspace{20mm}\\
			
			Pros:\vspace{10mm}\\
			Cons:
			
			
		\end{multicols}
		\btVFill
		\tiny{Text: Theory and Design for Mechanical Measurements}
	\end{frame}

\section{\sectiontitleII}	

% Section II - Frame I
\begin{frame}[label=sectionII] \scriptsize
\frametitle{\sectiontitleII}
\bigskip

{\RD Electromagnetic Interference}:  An electromagnetic disturbance that interrupts, obstructs, or otherwise degrades or limits the effective performance of electronics/electrical equipment.\vspace{5mm}\\

A {\it combination} of naturally occuring and human made sources of interference is always present. The total EMI affecting a system is determined by the local conditions as well as global environmental influences. \vspace{5mm}\\


Sources of EMI:
\begin{itemize}

	\item Televsion Transimission, Celular Networks, AM FM Radio 
	\item Lightening Storms, Solar Activity 
	\item Power Transmission Lines	
	\item Electronic Devices such as power supplies, motors, welders
 
	\item Intentional (weaponized) EMI	

\end{itemize}


\btVFill
\tiny{\href{https://csrc.nist.gov/glossary/term/electromagnetic_interference}{NIST}}		

\end{frame}

\section{\sectiontitleIII}	

% Section III - Frame I
\begin{frame}[label=sectionIII] \small
\frametitle{\sectiontitleIII}
\bigskip

\begin{itemize}
	\item National Instruments \vspace{6mm}\\
	
	\item Measurement Computing  \vspace{6mm}\\
	
	\item dSPACE  \vspace{6mm}\\
	
	\item Arduino or other 
	
\end{itemize}

\btVFill
\tiny{some reference}		

\end{frame}
	
\section{\sectiontitleIV}	

% Section IV - Frame I
\begin{frame}[label=sectionIV] \small
\frametitle{\sectiontitleIV}
\bigskip

\begin{itemize}
	\item National Instruments \vspace{6mm}\\
	
	\item Measurement Computing  \vspace{6mm}\\
	
	\item dSPACE  \vspace{6mm}\\
	
	\item Arduino or other 
	
\end{itemize}


\btVFill
\tiny{some reference}		

\end{frame}

% Section IV - Frame II
\begin{frame}[label=sectionIV] \small
\frametitle{\sectiontitleIV}
\bigskip


\btVFill
\tiny{some reference}	
\end{frame}


\end{document}
