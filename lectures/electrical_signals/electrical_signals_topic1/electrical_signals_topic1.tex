% !TeX document-id = {3ffda977-020f-403a-a748-6559a1e64ed1}
% !TeX TXS-program:compile = txs:///pdflatex/[--shell-escape]
% % !TEX TS-program = xelatex
% !TEX encoding = UTF-8 Unicode


% Spring 2020 - Summer 2020 - Fall 2020 - SPring 2021
% Tristan Hill, May 07, 2020 - June 12, 2020 - July 08, 2020 - Novemeber 02, 2020 - March 28, 2021
% Module Name: - Electrical Signals
% Topic 1 - Introduction to Electrical Signals

\documentclass[fleqn]{beamer} % for presentation (has nav buttons at bottom)

\usepackage{/home/thill/Documents/lectures/measurements_lectures/measurements_lectures}


\author{ME3023 - Measurements in Mechanical Systems} % original formatting from Mike Renfro, September 21, 2004

\newcommand{\MNUM}{9\hspace{2mm}} % Module number - The module number should be phased out...
\newcommand{\TNUM}{1\hspace{2mm}} % Topic number - Topic numbers are going to stay
\newcommand{\moduletitle}{Electrical Signals}
\newcommand{\topictitle}{Classification of Signals  } 

\newcommand{\sectiontitleI}{Introduction to Signal Concepts}
\newcommand{\sectiontitleII}{Analog, Discrete, or Digital}
\newcommand{\sectiontitleIII}{Static or Dynamic}
\newcommand{\sectiontitleIV}{Deterministic or Non-Deterministic}

\newcommand{\btVFill}{\vskip0pt plus 1filll}


% custom box
\newsavebox{\mybox}

\title{Lecture Module - \moduletitle}

\date{Mechanical Engineering\vspc Tennessee Technological University}

\begin{document}

\lstset{language=MATLAB,basicstyle=\ttfamily\small,showstringspaces=false}

\frame{\titlepage \center\begin{framed}\Large \textbf{Topic \TNUM - \topictitle}\end{framed} \vspace{5mm}}

% Section 0: Outline
\frame{
\large \textbf{Topic \TNUM - \topictitle} \vspace{3mm}\\

\begin{itemize}

	\item \sectiontitleI    \vspc % Section I
	\item \sectiontitleII 	\vspc % Section II
	\item \sectiontitleIII 	\vspc %Section III
	\item \sectiontitleIV 	\vspc %Section IV
%	\item \sectiontitleV 	\vspc %Section IV

\end{itemize}

}

% Section I:
\section{\sectiontitleI}

	% Section I - Frame I:
	\begin{frame}[label=sectionI] \small
		\frametitle{\sectiontitleI}
		
		\begin{multicols}{2}
		{\PR Signal}, {\RD Amplitude}, and {\BL Frequency}\vspace{2mm}\\
		\includegraphics[scale=.3]{amplitude_frequency.png} 
		%\includegraphics[scale=.25]{unit_circle.png}
		
		{\it The shape and form of a {\PR signal} are often referred to as its {\PN waveform}.
		The {\PN waveform} contains information about the magnitude and {\RD amplitude}, which indicate the size of
		the input quantity, and the {\BL frequency}, which indicates the way the {\PR signal} changes in time. }\vspace{5mm}\\
		\end{multicols}
		\btVFill
		\tiny{Text: Theory and Design for Mechanical Measurements}
	\end{frame}

	% Section I - Frame II
	\begin{frame}[label=sectionI] \small
		\frametitle{\sectiontitleI}    				
		\bigskip
		
		{\it A {\PR signal} is the physical information about a measured variable being transmitted
		between a process and the measurement system, between the stages of a measurement system, or as
		the output from a measurement system. } \\
	
		\includegraphics[scale=.07]{computer_motherboard.jpg} 
		\includegraphics[scale=.06]{curiosity_on_mars.jpg} 
		\includegraphics[scale=.22]{tesla_underhood.jpg}
		
		\btVFill
		\tiny{Images: Wikipedia}	
	\end{frame}


\section{\sectiontitleII}	
	% Section II - Frame I
	\begin{frame}[label=sectionII] \small
		\frametitle{\sectiontitleII}
		

	    \begin{itemize}
			\item \textbf{Analog Signal- magnitude is continuous in time }  \vspace{3mm} \\
			\item \textbf{Discrete Time Signal- magnitude at points in time}  \vspace{3mm} \\
			\textbf{ \hspace*{15mm} - sampling at repeated time intervals}  \vspace{3mm} \\
			\item \textbf{Digital Signal- exists in discrete points in time}  \vspace{3mm} \\
			\textbf{ \hspace*{15mm} - magnitude is also discrete}  \vspace{3mm} \\
		\end{itemize}
		
		
	\end{frame}

	% Section II - Frame II
	\begin{frame} \small
		\frametitle{\sectiontitleII}
		\bigskip    
		
		{\it {\RD Analog} describes a signal that is
		continuous in time. Because physical variables tend to be continuous, an analog signal provides a
		ready representation of their time-dependent behavior. }

		\vspace{30mm}
		Examples: voltage in a circuit
%	\includegraphics[scale=.25]{topic2_histogram_fig1}
		\btVFill
		\tiny{Text: Theory and Design for Mechanical Measurements}	
	\end{frame}
	
	% Section II - Frame III
	\begin{frame} \small
		\frametitle{\sectiontitleII}    
		\bigskip  
			
		{\it ...a {\GR discrete time} signal, for which information about the
		magnitude of the signal is available only at discrete points in time. A discrete time signal usually
		results from the sampling of a continuous variable at repeated finite time intervals. }

%	\includegraphics[scale=.25]{topic2_histogram_fig2}
		\vspace{30mm}
		Examples:
		\btVFill
		\tiny{Text: Theory and Design for Mechanical Measurements}	
	\end{frame}
	
	% Section II - Frame IV
	\begin{frame} \small
		\frametitle{\sectiontitleII}    
		\bigskip 
			{\it A {\BL digital} signal has two important characteristics. First, a digital signal
			exists at discrete values in time, like a discrete time signal. Second, the magnitude of a digital signal
			is discrete, determined by a process known as {\PR quantization} at each discrete point in time. }

%	\includegraphics[scale=.25]{topic2_histogram_fig3}
		\vspace{30mm}
		Examples:
		\btVFill
		\tiny{Text: Theory and Design for Mechanical Measurements}	
	\end{frame}
	
\section{\sectiontitleIII}	
% Section III - Frame I
\begin{frame}[label=sectionIII] \small
\frametitle{\sectiontitleIII}
\bigskip

Signals may be characterized as either static or
dynamic. A static signal does not vary with time.

A dynamic signal is defined as a time-dependent signal. In general, dynamic signal waveforms,
y(t), may be classified as shown in Table 2.1.
	
\btVFill
\tiny{Text: Theory and Design for Mechanical Measurements}		
\end{frame}

\section{\sectiontitleIV}	
% Section IV - Frame I
\begin{frame}[label=sectionIV] \small
\frametitle{\sectiontitleIV}
\bigskip

A deterministic signal varies in time in a predictable
manner, such as a sine wave, a step function, or a ramp function, as shown in Figure 2.5. A signal is
steady periodic if the variation of the magnitude of the signal repeats at regular intervals in time.

Also described in Figure 2.5 is a non-deterministic signal that has no discernible pattern of
repetition. A non-deterministic signal cannot be prescribed before it occurs, although certain characteristics of the signal may be known in advance.

\btVFill
\tiny{Text: Theory and Design for Mechanical Measurements}	
\end{frame}

% Section IV - Frame II
\begin{frame}[label=sectionIV] \small
\frametitle{\sectiontitleIV}
\bigskip

\includegraphics[scale=.15]{table_2_1.png}

\btVFill
\tiny{Table 2.1 : Theory and Design for Mechanical Measurements}	
\end{frame}

\end{document}
