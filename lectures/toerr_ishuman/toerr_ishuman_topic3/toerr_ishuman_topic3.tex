% Lecture Template for ME3023 -  Measurements in Mechanical Systems - Tennessee Technological University
% Spring 2020 - Summer 2020 - Fall 2020 - Spring 2021 - Summer 2021
% Tristan Hill, May 07, 2020 - June 12, 2020 - July 08, 2020 - Novemeber 02, 2020 - March 28, 2021 - May 25, 2021
% Module Name: To Err is Human
% Topic 3 - Repeatability and Testing

\documentclass[fleqn]{beamer} % for presentation (has nav buttons at bottom)

\usepackage{/home/thill/Documents/lectures/measurements_lectures/measurements_lectures}

\author{ME3023 - Measurements in Mechanical Systems}

\newcommand{\MNUM}{2\hspace{2mm}} % Module number
\newcommand{\TNUM}{3\hspace{2mm}} % Topic number 
\newcommand{\moduletitle}{To Err is Human}
\newcommand{\topictitle}{Repeatability and Testing} 

\newcommand{\sectiontitleI}{Instrument Repeatability}
\newcommand{\sectiontitleII}{Conditions for Repeatability}
\newcommand{\sectiontitleIII}{Reproducibility and Instrument Uncertainty}
%\newcommand{\sectiontitleIV}{Sample Uncertainty Data}

% custom box
\newsavebox{\mybox}

\title{Lecture Module - \moduletitle}

\date{Mechanical Engineering\vspc Tennessee Technological University}

\begin{document}
	
	\lstset{language=MATLAB,basicstyle=\ttfamily\small,showstringspaces=false}
	
	\frame{\titlepage \center\begin{framed}\Large \textbf{Topic \TNUM - \topictitle}\end{framed} \vspace{5mm}}

% Section 0: Outline
\begin{frame}

\large \textbf{Topic \TNUM - \topictitle} \vspace{3mm}\\

\begin{itemize}

	\item \hyperlink{sectionI}{\sectiontitleI} \vspc % Section I
	\item \hyperlink{sectionII}{\sectiontitleII} \vspc % Section II
	\item \hyperlink{sectionIII}{\sectiontitleIII} \vspc %Section III

\end{itemize}

\end{frame}


% Section 1
\section{\sectiontitleI}

\begin{frame}[label=sectionI]
\frametitle{\sectiontitleI}

``The ability of a measurement system to indicate the same value on repeated but independent
application of the same input provides a measure of the instrument {\PN repeatability}. Specific claims of
{\PN repeatability} are based on multiple calibration tests (replication) performed within a given lab on the
particular unit.'' \vspc

\begin{framed}\hspace{10mm}\scalebox{1}{$\%u_{R_{max}}=\frac{2s_x}{r_0}\times100$}\end{framed}

\vspace{10mm}

{\tiny Text: Theory and Design of Mech. Meas.}

\end{frame}


\section{\sectiontitleII}

\begin{frame}[label=sectionII]
\frametitle{\sectiontitleII}

The following conditions need to be fulfilled in the establishment of repeatability:
\begin{itemize}

\item the same experimental tools
\item the same observer
\item the same measuring instrument, used under the same conditions
\item the same location
\item repetition over a short period of time.
\item same objectives


\end{itemize}
\vspace{5mm}
{\tiny Text: \href{https://en.wikipedia.org/wiki/Repeatability}{Wikipedia(NIST)} }

\end{frame}

% Section 3
\section{\sectiontitleIII}

\begin{frame}[label=sectionIII]
\frametitle{\sectiontitleIII}

``The term {\GR reproducibility}, when reported in instrument specifications, refers to the closeness of
agreement in results obtained from duplicate tests carried out under similar conditions of
measurement ... \vspcc

... The term {\PR instrument precision}, when reported in instrument specifications, refers to a random
uncertainty based on the results of separate repeatability tests.'' \vspace{10mm} \\

{\tiny Text: Theory and Design of Mech. Meas.}

\end{frame}

\end{document}





