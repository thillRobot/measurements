
% Lecture Template for ME3023 -  Measurements in Mechanical Systems - Tennessee Technological University
% Spring 2020 - Summer 2020 - Fall 2020 - Spring 2021 - Summer 2021
% Tristan Hill, May 07, 2020 - June 12, 2020 - July 08, 2020 - Novemeber 02, 2020 - March 28, 2021 - May 25, 2021
% Module Name: - Introduction
% Topic 2 - Types of Variables

\documentclass[fleqn]{beamer} % for presentation (has nav buttons at bottom)

\usepackage{/home/thill/Documents/lectures/measurements_lectures/measurements_lectures}

\author{ME3023 - Measurements in Mechanical Systems} % original formatting from Mike Renfro, September 21, 2004

\newcommand{\TNUM}{2\hspace{2mm}} % Topic number 
\newcommand{\moduletitle}{Introduction}
\newcommand{\topictitle}{Types of Variables}

\newcommand{\sectiontitleI}{Measured Variable}
\newcommand{\sectiontitleII}{Independent and Dependent Variables}
\newcommand{\sectiontitleIII}{Controlled Variables and Parameters}
\newcommand{\sectiontitleIV}{Extraneous Variables}
\newcommand{\sectiontitleV}{Engineering Example}

% custom box
\newsavebox{\mybox}

\title{Lecture Module - \moduletitle}

\date{Mechanical Engineering\vspc Tennessee Technological University}


\begin{document}

\lstset{language=MATLAB,basicstyle=\ttfamily\small,showstringspaces=false}

\frame{\titlepage \center\begin{framed}\Large \textbf{Topic \TNUM - \topictitle}\end{framed} \vspace{5mm}}

% Section 0: Outline
\begin{frame}

\large \textbf{Topic \TNUM - \topictitle} \vspace{3mm}\\

\begin{itemize}

	\item \hyperlink{sectionI}{\sectiontitleI} \vspc % Section I
	\item \hyperlink{sectionII}{\sectiontitleII} \vspc % Section II
	\item \hyperlink{sectionIII}{\sectiontitleIII} \vspc %Section III
	\item \hyperlink{sectionIV}{\sectiontitleIV} \vspc %Section IV
	\item \hyperlink{sectionV}\sectiontitleV 	\vspc %Section IV

\end{itemize}

\end{frame}

% Section 1
\section{\sectiontitleI}
\begin{frame}[label=sectionI]
\frametitle{\sectiontitleI}

\large{``A {\BL measurement} is an act of assigning a specific value to a physical variable. That physical variable
is the {\GR measured variable}.''} \vspc
{\tiny Text: Theory and Design of Mech. Meas.}

\end{frame}

% Section 2
\section{\sectiontitleII}
\begin{frame}[label=sectionII]
\frametitle{Independent and Dependent Variables}

{``If a change in one variable will not affect the value of some other variable, the
two are considered independent of each other. A variable that can be changed independently of other
variables is known as an \hspcuu \hspcc \hspcuu. A variable that is affected by changes in one or more
other variables is known as a \hspcuu \hspcc \hspcuu. Normally, the variable that we measure depends on
the value of the variables that control the process.''} \vspc
{\tiny Text: Theory and Design of Mech. Meas.}

\end{frame}

% Section 3
\section{\sectiontitleIII}
\begin{frame}[label=sectionIII]
\frametitle{\sectiontitleIII}

{``A variable is \hspcuu if it can be held at a constant value
or at some prescribed condition during a measurement... ...complete control of a variable is not usually
possible. We use the adjective \hspcuu to refer to a variable that can be held as prescribed, at
least in a nominal sense... \vspc
...we define a \hspcuu as a functional grouping of variables. For example, a moment of inertia or a Reynolds number... ...A \hspcuu that has an effect on the behavior of the measured variable is called a control parameter....''} \vspc
{\tiny Text: Theory and Design of Mech. Meas.}

\end{frame}

% Section 4
\section{\sectiontitleIV}
\begin{frame}[label=sectionIV]
\frametitle{\sectiontitleIV}

{``Variables that are not or cannot be \hspcuu during measurement but that affect the value of the
variable measured are called \hspcuu \hspcc \hspcuu. Their influence can confuse the clear relation
between cause and effect in a measurement... ...The effects due to \hspcuu \hspcc \hspcuu can take the form of signals superimposed
onto the measured signal with such forms as {\PR noise} and drift.''} \vspc
{\tiny Text: Theory and Design of Mech. Meas.}

\end{frame}

% Section 5
\section{\sectiontitleV}
\begin{frame}[label=sectionV]
\frametitle{\sectiontitleV}


	\textbf{ SHARP I.R. Ranger - Distance Sensor} \vspc
	\includegraphics[scale=0.5]{proximity_sensor.jpg} \hspace{20mm}\includegraphics[scale=0.20]{sharp_ranger_circuit.png} \vspc
  {\tiny Image, More Info: \href{https://en.wikipedia.org/wiki/Proximity_sensor}{Wikipedia} }\hspace{40mm} {\tiny Image, More Info: \href{https://en.wikipedia.org/wiki/Position_sensitive_device}{Wikipedia} }

\end{frame}

\begin{frame}
\frametitle{Engineering Example}


Consider the IR distance ranger, name at least one physical variable for each of the following categories. 

\begin{itemize}
	\item Measured Variable \vspc 
	\item Independent Variable \vspc
	\item Dependent Variables  \vspc 
	\item Controlled Variables  \vspc 
	\item Extraneous Variables  \vspc 
\end{itemize}

\end{frame}

\end{document}





