
% Lecture Template for ME3023 -  Measurements in Mechanical Systems - Tennessee Technological University
% Spring 2020 - Summer 2020 - Fall 2020 - Spring 2021 - Summer 2021
% Tristan Hill, May 07, 2020 - June 12, 2020 - July 08, 2020 - Novemeber 02, 2020 - March 28, 2021 - May 25, 2021
% Module Name: - Introduction
% Topic 3 - Experimental Test Plan

\documentclass[fleqn]{beamer} % for presentation (has nav buttons at bottom)

\usepackage{/home/thill/Documents/lectures/measurements_lectures/measurements_lectures}

\author{ME3023 - Measurements in Mechanical Systems} % original formatting from Mike Renfro, September 21, 2004

\newcommand{\TNUM}{3\hspace{2mm}} % Topic number 
\newcommand{\moduletitle}{Introduction}
\newcommand{\topictitle}{Experimental Test Plan}

\newcommand{\sectiontitleI}{Parameter Design Plan}
\newcommand{\sectiontitleII}{System and Tolerance Design Plan}
\newcommand{\sectiontitleIII}{Data Reduction Design Plan}
\newcommand{\sectiontitleIV}{Experimental Design Strategies }


% custom box
\newsavebox{\mybox}

\title{Lecture Module - \moduletitle}

\date{Mechanical Engineering\vspc Tennessee Technological University}


\begin{document}
	
\lstset{language=MATLAB,basicstyle=\ttfamily\small,showstringspaces=false}
	
\frame{\titlepage \center\begin{framed}\Large \textbf{Topic \TNUM - \topictitle}\end{framed} \vspace{5mm}}

% Section 0: Outline
\begin{frame}

\large \textbf{Topic \TNUM - \topictitle} \vspace{3mm}\\

\begin{itemize}

	\item \hyperlink{sectionI}{\sectiontitleI} \vspc % Section I
	\item \hyperlink{sectionII}{\sectiontitleII} \vspc % Section II
	\item \hyperlink{sectionIII}{\sectiontitleIII} \vspc %Section III
	\item \hyperlink{sectionIV}{\sectiontitleIV} \vspc %Section IV

\end{itemize}

\end{frame}

% Section 1
\section{\sectiontitleI}

\begin{frame}[label=sectionI]
\frametitle{\sectiontitleI}

{\BL Parameter Design Plan}: Determine the test objective and identify the process variables and
parameters and a means for their control. \vspc
\underline{Ask}: \begin{itemize}
	\item What question am I trying to answer?
	\item What needs to be measured?
	\item What variables and parameters will affect my results?
\end{itemize} 

{\tiny Text: Theory and Design of Mech. Meas.}
\end{frame}

% Section 2
\section{\sectiontitleII}

\begin{frame}[label=sectionII]
\frametitle{\sectiontitleII}


{\PR System and Tolerance Design Plan}: Select a measurement technique, equipment, and
test procedure based on some preconceived tolerance limits for error. \vspc
\underline{Ask}:
\begin{itemize}
	\item In what ways can I do the measurement?
	\item How good do the results need to be to answer my
	question? 
\end{itemize} 

{\tiny Text: Theory and Design of Mech. Meas.}
\end{frame}

% Section 3
\section{\sectiontitleIII}

\begin{frame}[label=sectionIII]
\frametitle{\sectiontitleIII}

{\GR Data Reduction Design Plan}: Plan how to analyze, present, and use the anticipated data.\vspc
\underline{Ask}:
\begin{itemize}
	\item How will I interpret the resulting data?
	\item How will I use the data to answer my question?
	\item How good is my answer? 
	\item Does my answer make sense?
\end{itemize} 

{\tiny Text: Theory and Design of Mech. Meas.}
\end{frame}

% Section 4
\section{\sectiontitleIV}

\begin{frame}[label=sectionIV]
\frametitle{\sectiontitleIV}


\begin{itemize}
	\item {\BL Randomized} Tests \vspccc% Section 1
	\item {\GR Repetition} and {\PR Replication}. \vspccc% Section 1
	\item {\BR Concomitant} Methods \vspccc
\end{itemize}

\end{frame}


\end{document}





