% 
% topic Template for ME3023 - Measurements in Mechanincal Systems - Tennessee Technological University
%
% Spring 2020 - Summer 2020
% Tristan Hill, May 31, 2020
% Introduction - Topic 3 - Experimental Test Plan
%

%\documentclass{beamer}                         % for presentation (has nav buttons at bottom)
\documentclass[handout]{beamer}  % for handout 
\usepackage{beamerthemesplit}
\usepackage{amsmath}
\usepackage{listings}
\usepackage{multicol}
\usepackage{framed}

\beamertemplateballitem

\definecolor{TTUpurple}{rgb}{0.3098, 0.1607, 0.5176} % TTU Purple (primary)
\definecolor{TTUgold}{rgb}{1.0000, 0.8666, 0.0000} % TTU Gold (primary)

\setbeamercolor{palette primary}{bg=TTUpurple,fg=TTUgold}
\setbeamercolor{palette secondary}{bg=black,fg=TTUgold}
\setbeamercolor{palette tertiary}{bg=black,fg=TTUpurple}
\setbeamercolor{palette quaternary}{bg=TTUgold,fg=black}
\setbeamercolor{structure}{fg=TTUpurple} % itemize, enumerate, etc
\setbeamercolor{section in toc}{fg=TTUpurple} % TOC sections

% custom colors 
\definecolor{mygray}{rgb}{.6, .6, .6}
\definecolor{mypurple}{rgb}{0.6,0.1961,0.8}
\definecolor{mybrown}{rgb}{0.5451,0.2706,0.0745}
\definecolor{mygreen}{rgb}{0, .39, 0}

% color commands
\newcommand{\R}{\color{red}}
\newcommand{\B}{\color{blue}}
\newcommand{\BR}{\color{mybrown}}
\newcommand{\K}{\color{black}}
\newcommand{\G}{\color{mygreen}}
\newcommand{\PR}{\color{mypurple}}
%\usefonttheme{professionalfonts}

\newcommand{\Lagr}{\mathcal{L}} % lagrangian

\newcommand{\vspccc}{\vspace{6mm}\\} % large vertical space
\newcommand{\vspcc}{\vspace{4mm}\\}   % medium vertical space
\newcommand{\vspc}{\vspace{2mm}\\}     % small vertical space

\newcommand{\hspcccc}{\hspace{10mm}} % large horizontal space
\newcommand{\hspccc}{\hspace{6mm}} % large horizontal space
\newcommand{\hspcc}{\hspace{4mm}}   % medium horizontal space
\newcommand{\hspc}{\hspace{2mm}}     % small horizontal space


\author{ME3023 - Measurements in Mechanical Systems} % original formatting from Mike Renfro, September 21, 2004

\newcommand{\MNUM}{1\hspace{2mm}} % Module number
\newcommand{\TNUM}{3\hspace{2mm}} % Topic number 
\newcommand{\moduletitle}{Introduction }
\newcommand{\topictitle}{Experimental Test Plan } 

\title{Module \MNUM - \moduletitle}

\date{May 29, 2020}

\begin{document}

\lstset{language=MATLAB,basicstyle=\ttfamily\small,showstringspaces=false}

\frame{\titlepage \center\begin{framed}\Large \textbf{Topic \TNUM - \topictitle}\end{framed} \vspace{5mm}}

% Section 0: Outline

\frame{

\large \textbf{Topic \TNUM - \topictitle} \vspace{3mm}\\

\begin{itemize}
	\item Parameter Design Plan \vspc% Section 1
	\item System and Tolerance Design Plan\vspc% Section 2
	\item Data Reduction Design Plan \vspc%Section 3
	\item Experimental Design Strategies \vspc%Section 4
\end{itemize}

}

% Section 1
\section{Parameter design plan}

\frame{
\frametitle{Parameter design plan}

{\B Parameter Design Plan}: Determine the test objective and identify the process variables and
parameters and a means for their control. \vspc
\underline{Ask}: \vspace{20mm}\\%‘‘What question am I trying to answer? What needs to be measured?’’ ‘‘What variables and parameters will affect my results?’’

{\tiny Text: Theory and Design of Mech. Meas.}
}

% Section 2
\section{System and tolerance design plan.}

\frame{
\frametitle{System and tolerance design plan}


{\PR System and Tolerance Design Plan}: Select a measurement technique, equipment, and
test procedure based on some preconceived tolerance limits for error. \vspc
\underline{Ask}: \vspace{20mm}\\%‘‘In what ways can I do the measurement and how good do the results need to be to answer my question?’’

{\tiny Text: Theory and Design of Mech. Meas.}
}
% Section 3
\section{Data reduction design plan}

\frame{
\frametitle{Data reduction design plan}

{\G Data Reduction Design Plan}: Plan how to analyze, present, and use the anticipated data.\vspc
\underline{Ask}: \vspace{20mm}\\% ‘‘How will I interpret the resulting data? How will I use the data to answer my question? How good is my answer? Does my answer make sense?’’

{\tiny Text: Theory and Design of Mech. Meas.}
}
% Section 4
\section{Experimental Design Strategies}

\frame{
\frametitle{Experimental Design Strategies}


\begin{itemize}
	\item {\B Randomized} Tests \vspccc% Section 1
	\item {\G Repetition} and {\PR Replication}. \vspccc% Section 1
	\item {\BR Concomitant} Methods \vspccc
\end{itemize}

}


\end{document}





