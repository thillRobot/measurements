% Lecture Template for ME3023-001- Tristan Hill - Spring 2018 - Summer 2018
% 
% Measurements in Mechanical Systems

% Document settings
\documentclass[11pt]{article}
% \usepackage[margin=1in]{geometry}
\usepackage[left=1.5cm, right=1.5cm, top=2cm]{geometry}
\usepackage[pdftex]{graphicx}
\usepackage{multirow}
\usepackage{setspace}
\usepackage{hyperref}
\usepackage{color,soul}
\usepackage{fancyvrb}
\usepackage{framed}
\usepackage{wasysym}
\usepackage{multicol}

\usepackage[utf8]{inputenc}
\usepackage[english]{babel}
 


\pagestyle{plain}
\setlength\parindent{0pt}
\hypersetup{
    bookmarks=true,         % show bookmarks bar?
    unicode=false,          % non-Latin characters in Acrobat’s bookmarks
    pdftoolbar=true,        % show Acrobat’s toolbar?
    pdfmenubar=true,        % show Acrobat’s menu?
    pdffitwindow=false,     % window fit to page when opened
    pdfstartview={FitH},    % fits the width of the page to the window
    pdftitle={My title},    % title
    pdfauthor={Author},     % author
    pdfsubject={Subject},   % subject of the document
    pdfcreator={Creator},   % creator of the document
    pdfproducer={Producer}, % producer of the document
    pdfkeywords={keyword1} {key2} {key3}, % list of keywords
    pdfnewwindow=true,      % links in new window
    colorlinks=true,       % false: boxed links; true: colored links
    linkcolor=red,          % color of internal links (change box color with linkbordercolor)
    citecolor=green,        % color of links to bibliography
    filecolor=magenta,      % color of file links
    urlcolor=blue           % color of external links
}

% assignment number 
\newcommand{\NUM}{9} 
\newcommand{\VSpaceSize}{2mm} 
\newcommand{\HSpaceSize}{2mm} 

\definecolor{mygray}{rgb}{.6, .6, .6}
\definecolor{mypurple}{rgb}{0.6,0.1961,0.8}
\definecolor{mybrown}{rgb}{0.5451,0.2706,0.0745}
\definecolor{mygreen}{rgb}{0, .39, 0}

\newcommand{\R}{\color{red}}
\newcommand{\B}{\color{blue}}
\newcommand{\BR}{\color{mybrown}}
\newcommand{\K}{\color{black}}
\newcommand{\G}{\color{mygreen}}
\newcommand{\PR}{\color{mypurple}}

\setulcolor{red} 
\setstcolor{green} 
\sethlcolor{mygray} 

\setlength{\parindent}{4em}
\setlength{\parskip}{1em}
\renewcommand{\baselinestretch}{1.5}


\begin{document}

\textbf{ \Large Tennessee Technological University - Fall 2019} \vspace{3mm}\\
\textbf{ \Large ME3023 Laboratory Challenge \NUM\hspace{2mm} - MyDAQ with MATLAB} \\

\begin{itemize}	
	\item \textbf{ \large The National Instruments MyDAQ data acquisition hardware can be used in the MATLAB programming environment and the Data Acquisition Toolbox.} \\
	
	\item \textbf{ \large  This allows you to input and output signals directly to and from MATLAB.} \\
	
	\item \textbf{ \large This requires MATLAB and the ``NI myDAQ Support from MATLAB'' add on. The software is pre-installed on the lab computers but can be used with your free student license.} \\

	\item \textbf{ \large Tutorials for Mathworks:}\\
	\begin{itemize}
	\item \href{https://www.mathworks.com/help/daq/index.html}{Data Acquisition Toolbox}	
	
	\item \href{https://www.mathworks.com/help/daq/analog-input-and-output.html}{Analog Input and Output}	

	\item \href{https://www.mathworks.com/help/daq/digital-input-and-output.html}{Digital Input and Output} \\
	\end{itemize}	

	\item \textbf{ \large Challenge: Replicate a previous lab challenge using the MyDAQ with MATLAB system. Record the necessary signals and if possible generate any needed excitation signals using the MyDAQ.}\\
	\end{itemize}	

	

\end{document}



