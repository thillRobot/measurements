% 
% Lecture Template for ME3023 -  Measurements in Mechanical Systems - Tennessee Technological University
%
% Spring 2020 - Summer 2020 - Fall 2020
% Tristan Hill, May 07, 2020 - June 12, 2020 - August 23, 2020
% Module 2 - To Err is Human
% Topic 3 - Repeatability and Testing
% 
%

\documentclass{beamer}                         % for presentation (has nav buttons at bottom)
%\documentclass[handout]{beamer}  % for handout 
\usepackage{beamerthemesplit}
\usepackage{amsmath}
\usepackage{listings}
\usepackage{multicol}
\usepackage{framed}

\beamertemplateballitem 


\definecolor{TTUpurple}{rgb}{0.3098, 0.1607, 0.5176} % TTU Purple (primary)
\definecolor{TTUgold}{rgb}{1.0000, 0.8666, 0.0000} % TTU Gold (primary)

\setbeamercolor{palette primary}{bg=TTUpurple,fg=TTUgold}
\setbeamercolor{palette secondary}{bg=black,fg=TTUgold}
\setbeamercolor{palette tertiary}{bg=black,fg=TTUpurple}
\setbeamercolor{palette quaternary}{bg=TTUgold,fg=black}
\setbeamercolor{structure}{fg=TTUpurple} % itemize, enumerate, etc
\setbeamercolor{section in toc}{fg=TTUpurple} % TOC sections

% custom colors
\definecolor{TTUpurple}{rgb}{0.3098, 0.1607, 0.5176} % TTU Purple (primary)
\definecolor{TTUgold}{rgb}{1.0000, 0.8666, 0.0000} % TTU Gold (primary) 
\definecolor{mygray}{rgb}{.6, .6, .6}
\definecolor{mypurple}{rgb}{0.6,0.1961,0.8}
\definecolor{mybrown}{rgb}{0.5451,0.2706,0.0745}
\definecolor{mygreen}{rgb}{0, .39, 0}
\definecolor{mypink}{rgb}{0.9960, 0, 0.9960}

% color commands
\newcommand{\R}{\color{red}}
\newcommand{\B}{\color{blue}}
\newcommand{\BR}{\color{mybrown}}
\newcommand{\K}{\color{black}}
\newcommand{\G}{\color{mygreen}}
\newcommand{\PR}{\color{mypurple}}
\newcommand{\PN}{\color{mypink}}
\newcommand{\OR}{\color{TTU}}
\newcommand{\GD}{\color{TTUgold}}


\setbeamercolor{palette primary}{bg=TTUpurple,fg=TTUgold}
\setbeamercolor{palette secondary}{bg=black,fg=TTUgold}
\setbeamercolor{palette tertiary}{bg=black,fg=TTUpurple}
\setbeamercolor{palette quaternary}{bg=TTUgold,fg=black}
\setbeamercolor{structure}{fg=TTUpurple} % itemize, enumerate, etc
\setbeamercolor{section in toc}{fg=TTUpurple} % TOC sections

%\usefonttheme{professionalfonts}

\newcommand{\Lagr}{\mathcal{L}} % lagrangian

\newcommand{\hspcu}{\underline{\hspace{25mm}} } % large horizontal space w underline
\newcommand{\vspccc}{\vspace{6mm}\\} % large vertical space
\newcommand{\vspcc}{\vspace{4mm}\\}   % medium vertical space
\newcommand{\vspc}{\vspace{2mm}\\}     % small vertical space

\newcommand{\hspcccc}{\hspace{10mm}} % large horizontal space
\newcommand{\hspccc}{\hspace{6mm}} % large horizontal space
\newcommand{\hspcc}{\hspace{4mm}}   % medium horizontal space
\newcommand{\hspc}{\hspace{2mm}}     % small horizontal space

\newcommand{\eqscl}{0. 9}     % small horizontal space


\author{ME3023 - Measurements in Mechanical Systems} % original formatting from Mike Renfro, September 21, 2004

\newcommand{\MNUM}{2\hspace{2mm}} % Module number
\newcommand{\TNUM}{3\hspace{2mm}} % Topic number 
\newcommand{\moduletitle}{To Err is Human}
\newcommand{\topictitle}{Repeatability and Testing} 

\newcommand{\sectiontitleI}{Instrument Repeatability}
\newcommand{\sectiontitleII}{Conditions for Repeatability}
\newcommand{\sectiontitleIII}{Reproducibility and Instrument Uncertainty}
%\newcommand{\sectiontitleIV}{Sample Uncertainty Data}

% custom box
\newsavebox{\mybox}

\title{Module \MNUM - \moduletitle}

\date{Mechanical Engineering\vspc Tennessee Technological University}

\begin{document}

\lstset{language=MATLAB,basicstyle=\ttfamily\small,showstringspaces=false}

\frame{\titlepage \center\begin{framed}\Large \textbf{Topic \TNUM - \topictitle}\end{framed} \vspace{5mm}}

% Section 0: Outline
\frame{

\large \textbf{Topic \TNUM - \topictitle} \vspace{3mm}\\

\begin{itemize}

	\item \sectiontitleI		\vspc % Section I
	\item \sectiontitleII 	\vspc % Section II
	\item \sectiontitleIII 	\vspc %Section III
	%\item \sectiontitleIV 	\vspc %Section IV

\end{itemize}

}

% Section 1
\section{\sectiontitleI}

\frame{
\frametitle{\sectiontitleI}

``The ability of a measurement system to indicate the same value on repeated but independent
application of the same input provides a measure of the instrument {\PN repeatability}. Specific claims of
{\PN repeatability} are based on multiple calibration tests (replication) performed within a given lab on the
particular unit.'' \vspc

\begin{framed}\hspace{10mm}\scalebox{1}{$\%u_{R_{max}}=\frac{2s_x}{r_0}\times100$}\end{framed}

\vspace{10mm}

{\tiny Text: Theory and Design of Mech. Meas.}
}


\section{\sectiontitleII}

\frame{
\frametitle{\sectiontitleII}

The following conditions need to be fulfilled in the establishment of repeatability:
\begin{itemize}

\item the same experimental tools
\item the same observer
\item the same measuring instrument, used under the same conditions
\item the same location
\item repetition over a short period of time.
\item same objectives


\end{itemize}
\vspace{5mm}
{\tiny Text: \href{https://en.wikipedia.org/wiki/Repeatability}{Wikipedia(NIST)} }
}

% Section 3
\section{\sectiontitleIII}

\frame{
\frametitle{\sectiontitleIII}

``The term {\G reproducibility}, when reported in instrument specifications, refers to the closeness of
agreement in results obtained from duplicate tests carried out under similar conditions of
measurement ... \vspcc

... The term {\PR instrument precision}, when reported in instrument specifications, refers to a random
uncertainty based on the results of separate repeatability tests.'' \vspace{10mm} \\

{\tiny Text: Theory and Design of Mech. Meas.}
}

%Section 3


\end{document}





